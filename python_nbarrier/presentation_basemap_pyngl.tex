%%%%%%%%%%%%%%%%%%%%%%%%%%%%%%%%%%%%%%%%%
% Beamer Presentation
% LaTeX Template
% Version 1.0 (10/11/12)
%
% This template has been downloaded from:
% http://www.LaTeXTemplates.com
%
% License:
% CC BY-NC-SA 3.0 (http://creativecommons.org/licenses/by-nc-sa/3.0/)
%
%%%%%%%%%%%%%%%%%%%%%%%%%%%%%%%%%%%%%%%%%

%----------------------------------------------------------------------------------------
%	PACKAGES AND THEMES
%----------------------------------------------------------------------------------------

\documentclass{beamer}

\mode<presentation> {

% The Beamer class comes with a number of default slide themes
% which change the colors and layouts of slides. Below this is a list
% of all the themes, uncomment each in turn to see what they look like.

%\usetheme{default}
%\usetheme{AnnArbor}
%\usetheme{Antibes}
%\usetheme{Bergen}
%\usetheme{Berkeley}
%\usetheme{Berlin}
%\usetheme{Boadilla}
%\usetheme{CambridgeUS}
%\usetheme{Copenhagen}
%\usetheme{Darmstadt}
%\usetheme{Dresden}
%\usetheme{Frankfurt}
%\usetheme{Goettingen}
%\usetheme{Hannover}
%\usetheme{Ilmenau}
%\usetheme{JuanLesPins}
%\usetheme{Luebeck}
%\usetheme{Madrid}
%\usetheme{Malmoe}
%\usetheme{Marburg}
%\usetheme{Montpellier}
%\usetheme{PaloAlto}
%\usetheme{Pittsburgh}
%\usetheme{Rochester}
%\usetheme{Singapore}
%\usetheme{Szeged}
\usetheme{Warsaw}

% As well as themes, the Beamer class has a number of color themes
% for any slide theme. Uncomment each of these in turn to see how it
% changes the colors of your current slide theme.

%\usecolortheme{albatross}
%\usecolortheme{beaver}
%\usecolortheme{beetle}
%\usecolortheme{crane}
%\usecolortheme{dolphin}
%\usecolortheme{dove}
%\usecolortheme{fly}
%\usecolortheme{lily}
%\usecolortheme{orchid}
%\usecolortheme{rose}
%\usecolortheme{seagull}
%\usecolortheme{seahorse}
%\usecolortheme{whale}
%\usecolortheme{wolverine}

%\setbeamertemplate{footline} % To remove the footer line in all slides uncomment this line
%\setbeamertemplate{footline}[page number] % To replace the footer line in all slides with a simple slide count uncomment this line

%\setbeamertemplate{navigation symbols}{} % To remove the navigation symbols from the bottom of all slides uncomment this line
}

\usepackage{graphicx} % Allows including images
\usepackage{booktabs} % Allows the use of \toprule, \midrule and \bottomrule in tables
\usepackage{listings}
\usepackage{color}
\usepackage{hyperref}
\usepackage[lofdepth,lotdepth]{subfig}

\definecolor{codegreen}{rgb}{0,0.6,0}
\definecolor{codegray}{rgb}{0.5,0.5,0.5}
\definecolor{codepurple}{rgb}{0.58,0,0.82}
\definecolor{backcolour}{rgb}{0.95,0.95,0.92}

\lstdefinestyle{mystyle}{
    backgroundcolor=\color{backcolour},   
    commentstyle=\color{codegreen},
    keywordstyle=\color{magenta},
    numberstyle=\tiny\color{codegray},
    stringstyle=\color{codepurple},
    %basicstyle=\footnotesize,
    basicstyle=\scriptsize\ttfamily,
    breakatwhitespace=false,         
    breaklines=true,                 
    captionpos=b,                    
    keepspaces=true,                 
    numbers=left,                    
    numbersep=5pt,                  
    showspaces=false,                
    showstringspaces=false,
    showtabs=false,                  
    tabsize=2,
    language=bash
}
 
\lstset{style=mystyle}

%----------------------------------------------------------------------------------------
%	TITLE PAGE
%----------------------------------------------------------------------------------------

\title[Spatial representation with Python]{Spatial representation with the Python Basemap and PyNGL libraries} % The short title appears at the bottom of every slide, the full title is only on the title page

\author{Nicolas Barrier} % Your name
\institute[UMR MARBEC] % 
{
UMR MARBEC \\ % Your institution for the title page
\medskip
\textit{nicolas.barrier@ird.fr} % Your email address
}
\date{\today} % Date, can be changed to a custom date

\hypersetup{
    colorlinks=true,
    linkcolor=white,
    urlcolor=blue,
} 

\begin{document}


\begin{frame}
\titlepage % Print the title page as the first slide
    \vspace{-1em}
\begin{center}
\includegraphics[height=2cm]{logo-marbec.png}
\hspace{1em}
\includegraphics[height=2cm]{logo_ird.png}
\end{center}
\end{frame}

\section{Basemap library}
\begin{frame}[fragile]
\frametitle{Description}
    The \href{https://matplotlib.org/basemap/}{Basemap} library is the mapping extension of the standard graphic library
    \footnotesize{
\begin{block}{Pros}
    \begin{itemize}
        \item{It is now quite easy to install by using \href{https://www.anaconda.com/}{Anaconda}}
        \item{Very simple to use}
        \item{Multiple map backgrounds can be used (from TIFF image for instance)}
        \item{Comes with usefull tool (distance calculations, etc.)}
    \end{itemize}
\end{block}
\begin{alertblock}{Cons}
    \begin{itemize}
        \item{Rendering is not always perfect}
        \item{No Masked Lambert Conformal projection}
        \item{Need to switch from geographical to map coordinates}
        \item{Need to navigate between Matplotlib and Basemap methods/functions}
    \end{itemize}
\end{alertblock}
}
\end{frame}

\begin{frame}[fragile]
\frametitle{Example}
\lstinputlisting[language=Python,basicstyle=\tiny\ttfamily, ]{progs/map_contour_quiver.py}
\end{frame}

\begin{frame}[fragile]
\frametitle{Example}
    \begin{center}
        \includegraphics[scale=0.8]{progs/contour_quiver.pdf}
    \end{center}
\end{frame}


\section{PyNGL library}

\begin{frame}[fragile]
\frametitle{Description}
    The NCAR Graphic library is a very powerfull library for graphics, especially mapping. It is available in Python through the \textcolor{blue}{\href{https://www.pyngl.ucar.edu/}{PyNGL}} library.
    \footnotesize{
\begin{block}{Pros}
    \begin{itemize}
        \item{It is now quite easy to install by using \href{https://www.anaconda.com/}{Anaconda}}
        \item{Very nice rendering}
        \item{The drawing of velocity fields is very easy (curlyvectors, etc.)}
        \item{Many examples are provided in the \href{https://www.pyngl.ucar.edu/Examples/gallery.shtml}{gallery}}
    \end{itemize}
\end{block}
\begin{alertblock}{Cons}
    \begin{itemize}
        \item{Not compatible with the default Python graphic library \href{https://matplotlib.org/}{Matplotlib}}
        \item{Heavy code! To use only for paper plots (not for working material)}
        \item{Only 256 colors can be used}
    \end{itemize}
\end{alertblock}
}
\end{frame}

\begin{frame}[fragile]
\frametitle{Simple map}
\lstinputlisting[language=Python,basicstyle=\tiny\ttfamily, ]{progs/simple_map_pyngl.py}
\end{frame}

\begin{frame}[fragile]
\frametitle{Simple map}
    \begin{center}
    \includegraphics[scale=0.3]{progs/conmasklc.png}
    \end{center}
\end{frame}

\begin{frame}[fragile]
\frametitle{More complicated example}
    366 lines of code later (cf. \verb+plot_currents_pyngl.py+)...
    \begin{center}
    \includegraphics[scale=0.28]{progs/oscar_traj_mean.png}
    \end{center}
\end{frame}

\begin{frame}[fragile]
\frametitle{Usefull links}
\scriptsize
    PyNgl Gallery: \url{https://www.pyngl.ucar.edu/Examples/gallery.shtml}\\
    Ncl Gallery: \url{https://www.ncl.ucar.edu/Applications/}\\
    Matplotlib Gallery: \url{https://matplotlib.org/basemap/users/examples.html}\\
    My Gallery: \url{www.nicolasbarrier.fr/gallery.html}\\
\end{frame}
\end{document}
